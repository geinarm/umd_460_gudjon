\documentclass[12pt]{article}
\usepackage{graphicx}
\usepackage{subcaption}
\usepackage[]{mcode}
\usepackage{mwe}
\usepackage{amsmath}
\usepackage[T1]{fontenc}
%\usepackage{lingmacros}
%\usepackage{tree-dvips}
%\usepackage{blindtext}
%\usepackage[utf8]{inputenc}

\renewcommand{\thesubsection}{\thesection.\alph{subsection}}

\begin{document}

\title{CMSC 460 - HW4}
\author{Gudjon Einar Magnusson}

\maketitle

%% 4.3
%Here is a cubic polynomial with three closely spaced real roots:
%p(x) = 816x^3 − 3835x^2 + 6000x − 3125.
\section{}

%(a) What are the exact roots of p?
\subsection{}

%(b) Plot p(x) for 1.43 ≤ x ≤ 1.71. Show the location of the three roots.
\subsection{}

%(c) Starting with x0 = 1.5, what does Newton’s method do?
\subsection{}

%(d) Starting with x0 = 1 and x1 = 2, what does the secant method do? 
\subsection{}

%(e) Starting with the interval [1, 2], what does bisection do?
\subsection{}

%(f) What is fzerotx(p,[1,2])? Why?
\subsection{}


%% 4.9. 
%Find the first ten positive values of x for which x = tan x.
\section{}


%% 4.16
%Utilities must avoid freezing water mains. If we assume uniform soil condi- tions, the temperature T(x,t) 
%at a distance x below the surface and time t after the beginning of a cold snap is given approximately by
%(T(x,t)−T_s)/(T_i-T_s) = erf(x / 2*sqrt(α*t))
%Here Ts is the constant surface temperature during the cold period, T_i is the initial soil temperature 
%before the cold snap, and α is the thermal conductivity of the soil. If x is measured in meters and t in 
%seconds, then α = 0.138 · 10−6 m2/s. Let T_i = 20◦ C, and T_s = −15◦ C, and recall that water freezes at 0◦ C. 
%Use fzerotx to determine how deep a water main should be buried so that it will not freeze until at least 
%60 days’ exposure under these conditions.
\section{} %4.16


\end{document}